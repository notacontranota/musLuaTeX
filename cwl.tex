\documentclass[a4paper]{article}
\usepackage{musluatex}
\usepackage{float}

\begin{document}
\renewcommand\figurename{Ejemplo}

El \becuadrotxt\ es perfecto.

El \becuadro\ es pluscuamperfecto.

El acorde \cifAm{d}{m7} es ultrautilizado.

\tachado{Te amo}, te odio, dame más.

El \grado{6} grado siempre se arrodilla ante el \grado{5}, pero el \grado{1} siempre gana.

La \negra\ se cree \blancab, pero es una \redonda\ y nada más.

\noindent \musLuaTeX\ es mi paquete favorito.
\begin{figure}[H]
\begin{center}
\begin{ly}
\relative c'{
  \tempo "Adagio" 4=50
  \time 3/4
  \key aes \major
  \partial 4 es\pp (
  aes2\< ) c4-.
  bes ( aes g )
  f2.\sf (
  es\pp\fermata ) \bar "|."
}
\end{ly}
\end{center}
\caption{\emph{De tónica a dominante.}}
\end{figure}

La dińamica \dinamica{pp} es pianísimo, y la \dinamica{sf} significa \emph{sforzato}.

El ritmo \ritmo{f2 f4 f f8-> f-. f-. f-.} es una muestra del uso del comando \texttt{\textbackslash ritmo\{\}}.

Ella tocó en el piano \teclado{0,3,6,9} y él se enamoró.

Gustavo Kantor camina en \compas{5}{4}, pero sus pasos son \dinamica{ppp}. Mientras tanto, silba \muses{\relative do'{\time 6/8 do8. do16 ( re8 ) mi4.->\fermata}}. Al silbar el \emph{mi} recordó la cuerda \cuerda{1}.

El Padre Nuestro, antes, era más lindo:

\gregoriano{(c3)<c>P</c>a(e)ter(f) nos(g)ter(g_',)}

\greannotation{Hymn.}
\greannotation[r]{VII}
\gregoriano{
Xa(a) b(b) c(c) d(d) e(e) f(f) g(g) h(h) i(i) j(j) k(k) l(l) m(m) (::),(,):(:)::(::);(;);1(;1)}

\gresetinitiallines{0}
\gresetclef{invisible}
\begin{tabular}{ll}
\textbf{Tipo} & \textbf{Notación}                                                             \\
\hline
Punctum       & \begin{minipage}[t]{6cm}
                \gregoriano{a(a) b(b) c(c) d(d) e(e) f(f) g(g) h(h) i(i) j(j) k(k) l(l) m(m)}
                \end{minipage}                                                                \\
Virga         & \begin{minipage}[t]{6cm}
                \gregoriano{ge(ge) if(if)}
                \end{minipage}                                                                \\
Pes           & \begin{minipage}[t]{6cm}
                \gregoriano{eg(eg) fi(fi)}
                \end{minipage}                                                                \\


\hline
\end{tabular}


\end{document}
